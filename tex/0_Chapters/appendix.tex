\section{Numerical Time Integration Schemes}
The algorithms discussed in section \ref{sec:timeintegration} will work for any given problem. However, for the sake of clarity, the following initial value problem will be used to describe the numerical implementation of the time integration schemes.
\begin{center}
$\dot{y} = \lambda (y - \sin t) + \cos t$
\end{center}

\subsection{Explicit Euler Method}
\begin{python}
def explicit_euler(h, n, f, y_init, t_init):
    # Setting Initial Conditions for the Initial Value Problem
    y_0 = y_init
    t_0 = t_init

    # Initializing Solution Arrays with Initial Values
    y_explicit = np.array(y_0)
    t_explicit = np.array(t_0)

    # Initializing Parameters for Time Integration
    y_next = y_0
    t_next = t_0

    # Marching Forward in Time
    for i in range(n):

        # Calculate Solution at Next Time Step
        y_next = y_next + h * (f(y_next, t_next))

        # Increment Time Step
        t_next = t_next + h

        # Store Values for Plotting
        y_explicit = np.append(y_explicit, y_next)
        t_explicit = np.append(t_explicit, t_next)

    return y_explicit, t_explicit
\end{python}


\subsection{Implicit Euler Method - Analytical Reformulation}
\begin{python}
def f_hty(h, f, t, y):
    # Reformulated Derivative Function in h, t and y
    # Needs to be formulated for the given IVP
    return h * (1 / (1 - h*lmda) * f(y, t))

def implicit_euler_analytical(h, n, f_hty, y_init, t_init):
    # Setting Initial Conditions for the Initial Value Problem
    y_0 = y_init
    t_0 = t_init
	
    # Initializing Solution Arrays
    y_implicit_1 = np.array(y_0)
    t_implicit_1 = np.array(t_0)

    # Initializing Parameters for Time Integration
    y_next = y_0
    t_next = t_0

    # Marching Forward in time
    for i in range(n):

        # Increment Time
        t_next = t_next + h 

        # Calculate Solution at Next Time Step
        y_next = 
         + f_hty(h, f, t_next, y_next)

        # Storing Values for Plotting
        y_implicit_1 = np.append(y_implicit_1, y_next)
        t_implicit_1 = np.append(t_implicit_1, t_next)

    return y_implicit_1, t_implicit_1
\end{python}

\subsection{Implicit Euler Method - Matrix Inversion}
\begin{python}
def implicit_euler_matrix(h, n, y_init, t_init):
    # Setting Initial Conditions for the Initial Value Problem
    y_0 = y_init
    t_0 = t_init

    # Initializing Solution Arrays with Intitial Values
    y_implicit_2 = np.array(y_0)
    t_implicit_2 = np.array(t_0)

    # Initializing Parameters for Solution
    t_next = t_0
    b = np.zeros(n)
    A = np.zeros(n*n).reshape(n, n)

    #Setting Up Vector b
    for i in range(n):
        t_next = t_next + h
        t_implicit_2 = np.append(t_implicit_2, t_next)
        b_n = -h * (lmda * math.sin(t_next) - math.cos(t_next))
        if i==0:
            b[i] = b_n + y_0
        else:
            b[i] = b_n


    # Setting Up Matrix A
    for i in range(n):
        A[i,i] = 1 - h * lmda
        if i > 0:
            A[i, i-1] = -1

    # Matrix Inversion
    A_inv = np.linalg.inv(A)

    # Calculating Solution Space
    # A * y = b -----> y = inv(A) * b
    y_sol = A_inv @ b

    # Adding Initial Value to Solution Space
    y_implicit_2 = np.append(y_implicit_2, y_sol)

    return y_implicit_2, t_implicit_2
\end{python}

\subsection{Implicit Euler Method - Fixed Point Iteration}
\begin{python}
def implicit_euler_iterative(h, n, f, y_init, t_init, maxitr, tol):
	# Setting Initial Conditions for the Initial Value Problem
    y_0 = y_init
    t_0 = t_init
    
    # Initializing Solution Arrays
    y_implicit_3 = np.array(y_0)
    t_implicit_3 = np.array(t_0)

    # Initializing Parameters for Time Integration
    y_next = y_0
    t_next = t_0
    convergence = True

    # Marching Forward in time
    for i in range(n):
        if (convergence):
            # Initialize Error for Convergence testing
            err = 1
            j = 0

            # Predictor - Explicit Euler
            y_next_iter = y_next + h * (f(y_next, t_next))

            # Increment Time
            t_next = t_next + h

            # Iterative Convergence using Corrector
            while(convergence):
                # Store previous iteration to calculate error
                y_prev_iter = y_next_iter

                # Corrector 
                y_next_iter = y_next + h * f(y_next_iter, t_next)


                # Calculate Error
                err = abs(y_next_iter - y_prev_iter)
                
                # Increment Iteration Counter
                j +=1

                # When solution converges 
                if(err < tol):
                    # Use converged Value for next Time Step
                    y_next = y_next_iter

                    # Store Values for Plotting
                    y_implicit_3 = np.append(y_implicit_3, y_next)
                    t_implicit_3 = np.append(t_implicit_3, t_next)
                    break

                # When solution diverges
                if(j > maxitr):
                    convergence = False
                   
    return y_implicit_3, t_implicit_3
\end{python}
\pagebreak[4]